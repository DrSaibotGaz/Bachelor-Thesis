\usepackage[utf8]{inputenc} % Richtiges anzeigen von Umlauten und quasi allen anderen Schriftzeichen
\usepackage[T1]{fontenc} % Wichtig für alles was mehr als ASCII verwendet
\usepackage{csquotes} % Schöne Anführungsstriche mit \enquote{Text}
\usepackage{amsmath} % Bessere und schönere mathematische Formeln
\usepackage{amsfonts} % Erweiterte Zeichensätze für mathematische Formeln
\usepackage{amssymb} % Spezielle mathematische Symbole.
\usepackage{array} % Matrizen in mathematischen Formeln
\usepackage{textcomp} % Für textmu und textohm etc. um im Fließtext keine Mathematik 

\usepackage[version=3]{mhchem} % Für Chemische Formeln
\usepackage{braket} % Für das quantenmechanische Bra-Ket

%\usepackage[showframe,paper=a4paper,margin=1in]{geometry} % Anzeigen der Seitenränder, nützlich für debugging. http://ctan.org/pkg/geometry

\usepackage[pdftex]{hyperref} % Links richtig anzeigen. Sowohl innerhalb des Dokuments (Fußzeilen, Formeln), als auch ins Internet
\usepackage[bottom]{footmisc} % Zwingt Fußnoten an das Ende der Seite

\usepackage[ % Biblatex für die Zitate und Referenzen
	backend=biber,
	hyperref=true
		]{biblatex}

\usepackage{xkeyval} % Erlaubt "Variablen" zu definieren, wird für Titelseite gebraucht
\usepackage{graphicx} % Wichtig für das Einbinden von Grafiken
\usepackage{caption}
\usepackage{subcaption} % Einbinden von mehreren Grafiken in einer figure
\usepackage{geometry} % Seitenränder und Seiteneigenschaften setzen

\newcommand{\writeIn}[1]{\usepackage[#1]{babel}} % Definiert einen neuen Befehl um die Sprache des Dokuments zu setzen

\usepackage[usenames,dvipsnames]{color} % Farben für den todo Befehl
\newcommand{\todo}[1]{{\color{Cerulean}(TODO: #1)}} % Einfach \todo{Text} verwenden!

\usepackage{dirtree} % Erlaubt das erstellen von Dateibäumen
% \dirtreecomment{Text} erstellt einen Kommentar zu dem Verzeichnis bzw. der Datei
\newcommand{\dirtreecomment}[1]{\dotfill{} \begin{minipage}[t]{0.5\textwidth}#1\end{minipage}}